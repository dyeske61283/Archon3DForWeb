% Meta-Informationen -----------------------------------------------------------
%   Definition von globalen Parametern, die im gesamten Dokument verwendet
%   werden können (z.B auf dem Deckblatt etc.).
%
%   ACHTUNG: Wenn die Texte Umlaute oder ein Esszet enthalten, muss der folgende
%            Befehl bereits an dieser Stelle aktiviert werden:
%            \usepackage[latin1]{inputenc}
% ------------------------------------------------------------------------------
\usepackage[utf8]{inputenc}

\newcommand{\titel}{Realisierung des Spieleklassikers "Archon"}
\newcommand{\untertitel}{mit 3D-- und Webtechnologien}
\newcommand{\art}{Bachelorarbeit}


\newcommand{\hochschule}{Technische Hochschule Nürnberg (Ohm)}
\newcommand{\hochschuletyp}{Technische Hochschule}
\newcommand{\hochschulename}{Georg-Simon-Ohm}
\newcommand{\fachgebiet}{Software-Engineering}
\newcommand{\studienbereich}{Elektro-- und Informationstechnik}
\newcommand{\autor}{Kevin Dyes}
\newcommand{\matrikelnr}{2694420} 

\newcommand{\erstgutachter}{Prof. Dr. Röttger}
\newcommand{\zweitgutachter}{Prof. Dr. Hopf}
\newcommand{\jahr}{2018}
\newcommand{\ort}{Nürnberg}
\newcommand{\logoHS}{LogoOhmHochschule.jpg}
