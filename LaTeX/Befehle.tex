% Eigene Befehle und typographische Auszeichnungen für diese
%\addtocontents{toc}{\protect\newpage} %Seitenumbruch in Inhaltsverzeichnis erzwingen

% einfaches Wechseln der Schrift, z.B.: \changefont{cmss}{sbc}{n}
%\newcommand{\changefont}[3]{\fontfamily{#1} \fontseries{#2} \fontshape{#3} \selectfont}

% Abkürzungen mit korrektem Leerraum 
\newcommand{\sa}{\mbox{s.\,a.\ }}
\newcommand{\ua}{\mbox{u.\,a.\ }}
\newcommand{\zB}{\mbox{z.\,B.\ }}
\newcommand{\dahe}{\mbox{d.\,h.\ }}
\newcommand{\Vgl}{Vgl.\ }
\newcommand{\vgl}{vgl.\ }
\newcommand{\bzw}{bzw.\ }
\newcommand{\uvm}{uvm.\ }
\newcommand{\evtl}{evtl.\ }
\newcommand{\ggf}{ggf.\ }

\newcommand{\abbildung}[1]{Abbildung~\ref{fig:#1}}

\newcommand{\bs}{$\backslash$}
%\newcommand{\bs}[1]{\boldsymbol{#1}}

% erzeugt ein Listenelement mit fetter Überschrift 
\newcommand{\itemd}[2]{\item{\textbf{#1}}\\{#2}}

% einige Befehle zum Zitieren --------------------------------------------------
\newcommand{\Zitat}[2][\empty]{\ifthenelse{\equal{#1}{\empty}}{\citep{#2}}{\citep[#1]{#2}}}

% zum Ausgeben von Autoren
\newcommand{\AutorName}[1]{\textsc{#1}}
\newcommand{\Autor}[1]{\AutorName{\citeauthor{#1}}}

% verschiedene Befehle um Wörter semantisch auszuzeichnen ----------------------
\newcommand{\NeuerBegriff}[1]{\textbf{#1}}
\newcommand{\Fachbegriff}[1]{\textit{#1}}

\newcommand{\Eingabe}[1]{\texttt{#1}}
%\newcommand{\Code}[1]{\texttt{#1}}
\newcommand{\Datei}[1]{\texttt{#1}}

\newcommand{\Datentyp}[1]{\textsf{#1}}
\newcommand{\XMLElement}[1]{\textsf{#1}}
\newcommand{\Webservice}[1]{\textsf{#1}}

% verschiedene Befehle um Wörter semantisch auszuzeichnen ----------------------
\newcommand{\Index}[1]{#1\index{#1}}

%\newcommand{\Quelle}{\footnotesize Quelle: }
\newcommand{\Quelle}[1]{\footnotesize{}Quelle: #1}
% im Index erscheinen alle gepunkteten Linien direkt übereinander
\renewcommand{\dotfill}{\leaders\hbox to2mm{\hss.\hss}\hskip 2mm plus1fill}
%\renewcommand{\dotfill}{\leaders\hbox to 5p1{\hss.\hss}\hfill}

% alternativ on-the-fly Konvertierung beim Übersetzungsvorgang wird externer Grafikkonverter aufgerufen
%\DeclareGraphicsRule{.jpg}{.eps}{}{jpeg2ps #1}