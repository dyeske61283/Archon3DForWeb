% Kopf- und Fußzeilen ----------------------------------------------------------
\pagestyle{scrheadings}
% Kopf- und Fußzeile auch auf Kapitelanfangsseiten
\renewcommand{\chapterpagestyle}{scrheadings}
\renewcommand*{\chapterpagestyle}{scrheadings}
\renewcommand{\indexpagestyle}{scrheadings}
\renewcommand*{\indexpagestyle}{scrheadings}
% Schriftform der Kopfzeile
\renewcommand{\headfont}{\normalfont}

% Kopfzeile
\ihead{	\large{\textsc{\art{}} - \textsc{\titel}}\\
		\small{\untertitel} \\[2ex]
		\textit{\headmark}}
\chead{}
\ohead{} % \includegraphics[scale=0.11]{\logoHS}\\\includegraphics[scale=0.15]{\logoFA}}
\setlength{\headheight}{20mm} % Höhe der Kopfzeile
% Kopfzeile über den Text hinaus verbreitern
%\setheadwidth[-5mm]{textwithmarginpar}
\setheadwidth[-5mm]{180mm}
\setheadsepline[170mm]{1.2pt} % Trennlinie unter Kopfzeile

% Fußzeile
\setfootwidth[-5mm]{170mm} 
\setfootsepline[170mm]{1.0pt} % Trennlinie \"uber Fusszeile
\ifoot{\footnotesize{\copyright\ \autor\ / \jahr}}
%\cfoot{\footnotesize{\textbf{\textcolor{red}{ - - ENTWURF - - ENTWURF - - ENTWURF - - }}}}

%\cfoot{}
\ofoot{\pagemark\ \\}

% Abstand einstellen zwischen den Chaptern
%\vspace*{2.3\baselineskip} = ORIGINAL
% Abstand vor Kapitelüberschriften: 1/3 der Satzspiegelhöhe}}
\renewcommand*{\chapterheadstartvskip}{\vspace*{-\topskip}} 
% Abstand nach Kapitelüberschriften: 3 Zeilen
\renewcommand*{\chapterheadendvskip}{\vspace{0\baselineskip}}

\makeatletter 
\renewcommand{\subsubsection}{\@startsection%
   {subsubsection}%       name
   {3}%                   level
   {0mm}%                 indent
   {-0.250\baselineskip}%   beforeskip
   {+0.01\baselineskip}%    afterskip
   {\normalsize\rmfamily\bfseries}}%     style
\makeatother


% Captions Setup ----------------------------------------------------------

\renewcaptionname{ngerman}{\figurename}{Abb.}
\renewcommand{\lstlistingname}{Quellcode}
% caption Setup
\renewcommand\theContinuedFloat{\alph{ContinuedFloat}}

% Verzeichnis Setup ----------------------------------------------------------
\makeatletter
\renewcommand{\l@figure}{\@dottedtocline{1}{1.5em}{3em}}
\renewcommand{\l@subfigure}{\@dottedtocline{1}{4.5em}{2.5em}}
\renewcommand{\l@table}{\@dottedtocline{1}{1.5em}{3em}}
\renewcommand{\l@lstlisting}{\@dottedtocline{1}{1.5em}{3em}}
\makeatother
% Verzeichnis Namen ----------------------------------------------------------
\renewcaptionname{ngerman}{\contentsname}{Inhaltsverzeichnis}
\renewcaptionname{ngerman}{\refname}{Literaturverzeichnis}
\renewcaptionname{ngerman}{\listfigurename}{Abbildungsverzeichnis}
\renewcaptionname{ngerman}{\listtablename}{Tabellenverzeichnis}
\renewcaptionname{ngerman}{\indexname}{Stichwortverzeichnis}
\renewcommand{\nomname}{Abk\"urzungsverzeichnis}
\renewcommand{\lstlistlistingname}{Quellcodeverzeichnis}
\indexprologue{Das Verzeichnis ist in Haupt- und Unterbegriffe gegliedert. Ist ein Stichwort nicht unter den Hauptbegriffen gelistet, so ist es womöglich als Untereintrag zu finden.}
% sonstige typographische Einstellungen ----------------------------------------
% erzeugt ein wenig mehr Platz hinter einem Punkt
\frenchspacing 

% Schusterjungen und Hurenkinder vermeiden
\clubpenalty = 10000
\widowpenalty = 10000 
\displaywidowpenalty = 10000

% Quellcode-Ausgabe formatieren
\lstset{numbers=left, numberstyle=\tiny, numbersep=5pt, breaklines=true}
\lstset{emph={square}, emphstyle=\color{red}, emph={[2]root,base}, emphstyle={[2]\color{blue}}}

% Fußnoten fortlaufend durchnummerieren
\counterwithout{footnote}{chapter}
% Fußnoten - Abstand nach der Nummer
\let\oldfootnote\footnote
\renewcommand\footnote[1]{\oldfootnote{\hspace{1ex}#1}}

\let\oldfootnotetext\footnotetext
\renewcommand\footnotetext[1]{\oldfootnotetext{\hspace{1ex}#1}}
%\setlength{\footnotesep}{2ex}

% Longtable einstellen
\setlength{\LTpre}{2ex}
\setlength{\LTpost}{0.0ex}


% Textboxen  mit TIKZ -----------------------------------------------------------------------------------
\tikzset{content/.style	={text width={1.0\textwidth}, draw=black}}
\tikzset{head/.style	={text width={0.21\textwidth}, above=0em of main.north west, xshift=+0.5em, yshift=-0.15em, line width=0.2em, anchor=south west, draw=white, double=black, double distance = 0.1em}}

\newcommand{\ERKENNTNISbox}[1]{
{\centering\begin{tikzpicture}[
every node/.style={rectangle,rounded corners, top color=white, bottom color=black!5,very thick, inner sep=0.5em, minimum size=1.1em,} ]
\node[content](main) at (0,0){\onehalfspacing #1};
\node[head](surround){\textbf{ERKENNTNIS:}};
\end{tikzpicture}}}

\newcommand{\MERKEbox}[1]{
{\centering\begin{tikzpicture}[
every node/.style={rectangle,rounded corners, top color=white, bottom color=black!5,very thick, inner sep=0.5em, minimum size=1.1em,} ]
	\vspace{-75pt}
\node[content](main) at (0,0){\onehalfspacing #1};
\node[head](surround){\textbf{MERKE:}};
\end{tikzpicture}}}

\newcommand{\TODObox}[1]{
{\centering\begin{tikzpicture}[
every node/.style={rectangle,rounded corners, top color=white, bottom color=red!50,very thick, inner sep=0.5em, minimum size=1.1em,} ]
\node[content](main) at (0,0){\onehalfspacing \textbf{#1}};
\node[head](surround){\textbf{ToDo BOX:}};
\end{tikzpicture}}}

% Programmablaufpläne mit TIKZ------------------------------------------------------------------------------
\tikzset{
	papSE/.style = {
		ellipse, draw, 
		align = center, 
		text width = 20 mm, 
		text badly centered,
		font=\ttfamily\footnotesize,
		%line width = 0.5,
		minimum width = 2.5em,
		minimum height = 2em,
		inner sep = 0.5em,
		top color=white,
		bottom color=green!20,
	},
	papDescision/.style = {
		diamond, draw, 
		%text width = 2em, 
		align = center, 
		text badly centered,
		font=\ttfamily\footnotesize,
		line width = 0.5,
		%minimum width  = 1.5em,
		%minimum height = 1.5em,
		inner sep=0.5em,
		top color=white,
		bottom color=yellow!20,
	},
	papData/.style = {
		trapezium, draw,
		align = left,
		%align = center,
		%text badly centered,
		trapezium left angle=70,
		trapezium right angle=110,
		font=\ttfamily\footnotesize,
		%line width = 0.5,
		minimum width  = 10.0em,
		%minimum height = 1.5em,
		%text width = 5.0em,
		inner sep = 0.5em,
		top color=white,
		bottom color=orange!20,
	},
	papInput/.style = {
		trapezium, draw,
		align = left,
		%align = center,
		%text badly centered,
		trapezium left angle=110,
		trapezium right angle=110,
		font=\ttfamily\footnotesize,
		%line width = 0.5,
		minimum width  = 10.0em,
		%minimum height = 1.5em,
		%text width = 5.0em,
		inner sep = 0.5em,
		top color=white,
		bottom color=gray!20,
	},
	papFunction/.style = {
		draw, rectangle split,
		rectangle split horizontal,
		rectangle split parts = 3,
		rectangle split empty part width=-10pt,
		align = left, %center,
		text badly centered,
		font=\ttfamily\footnotesize,
		%line width = 0.5,
		minimum width  = 10.0em,
		minimum height = 1.5em,
		%text width = 4.5 em, 
		inner sep = 0.5em,
		top color=white,
		bottom color=red!20,
	},
	papInstruction/.style = {
		rectangle, draw,
		%rounded corners,
		align = left,
		%align = center,
		%text badly centered,
		font=\ttfamily\footnotesize,
		line width = 0.5,
		minimum width  = 10.0em,
		minimum height = 1.5em,
		%text width = 6.0em, 
		inner sep = 0.5em,
		top color=white,
		bottom color=blue!20,
	},
	papLine/.style = {
		draw,
		font=\ttfamily\footnotesize,
		line width = 0.75,
	},
	papArrow/.style = {
		draw,
		-stealth,
		font=\ttfamily\footnotesize,
		line width = 0.75,
	},
	papDotted/.style = {
		draw,
		dotted,
		-stealth,
		font=\ttfamily\footnotesize,
		line width = 1.0,
	},
	papDottedLine/.style = {
		draw,
		dotted,
		font=\ttfamily\footnotesize,
		line width = 1.0,
	},
	papCircle/.style = {
		circle , draw, 
		align = center, 
		%text badly centered,
		%text width = 20 mm, 
		font=\ttfamily\footnotesize,
		line width = 0.5,
		%minimum width = 2em,
		%minimum height = 2em,
		inner sep = 2pt,
		top color=white,
		bottom color=white!20,
	},
}

% Styles für Codefragmente ------------------------------------------------------------------------------
\newcommand{\Code}[1]{\texttt{\textbf{\textsl{#1}}}}

\newcommand{\HTMLcode}[1]{\HTML\texttt{\textbf{\textsl{#1}}}}
\newcommand{\HTML}{\includegraphics[height=12pt]{logo_HTML5} }

\newcommand{\CSScode}[1]{\CSS\texttt{\textbf{\textsl{#1}}}}
\newcommand{\CSS}{\includegraphics[height=12pt]{logo_CSS3} }

\newcommand{\JScode}[1]{\JS\texttt{\textbf{\textsl{#1}}}}
\newcommand{\JS}{\includegraphics[height=12pt]{logo_JavaScript} }

\newcommand{\JQcode}[1]{\JQ\texttt{\textbf{\textsl{#1}}}}
\newcommand{\JQ}{\includegraphics[height=12pt]{logo_jQuery} }

\newcommand{\SCLcode}[1]{\SCL\texttt{\textbf{\textsl{#1}}}}
\newcommand{\SCL}{\includegraphics[height=12pt]{logo_SCL} }

\newcommand{\CScode}[1]{\CS\texttt{\textbf{\textsl{#1}}}}
\newcommand{\CS}{\includegraphics[height=12pt]{logo_Csharp} }

\newcommand{\CSharp}{\texttt{C\#}}

\newcommand{\RED}[1]{\textcolor{red}{\textbf{{\textsl{#1}}}}}

% Styles für Listings ------------------------------------------------------------------------------------
\definecolor{colCommentsSCL}{rgb}{0,0.5,0.5}
\definecolor{hellgelb}{rgb}{1,1,0.92}
\definecolor{hellblau}{rgb}{0.2,0.6,0.6}
\definecolor{cssKey}{rgb}{0.0,0.0,1.0}
\definecolor{cssId}{rgb}{0,0.5,1.0}
\definecolor{cssClass}{rgb}{1.0,0.0,0.0}
\definecolor{cssPseudo}{rgb}{1.0,0.5,0.0}
\definecolor{css}{rgb}{0.5,0.5,1.0}
\definecolor{colKeys}{rgb}{0,0,1}
\definecolor{colIdentifier}{rgb}{0,0,0}
\definecolor{colComments}{rgb}{0,0.5,0}
\definecolor{colString}{rgb}{0,0,0.7}
\definecolor{colName}{rgb}{0.6,0.3,0.1}

\lstset{
    %extendedchars=true,
    inputencoding=utf8,
    float=hbp,
    captionpos=b, %bottom, %top
    basicstyle=\ttfamily\color{black}\small\smaller,
    columns=flexible,keepspaces,
    tabsize=4,
    frameround=fttt,
    frame=trBL, %single,
    extendedchars=true,
    showstringspaces=false,
    showspaces=false,
    showlines=true,
    numbers=left,                    % where to put the line-numbers; possible values are (none, left, right)
    numberstyle=\tiny,
    numbersep = 15pt,
    breaklines=true,
    backgroundcolor=\color{hellgelb!75},
    breakautoindent=true,
    morecomment=[l]{//},
    morecomment=[s]{/*}{*/},
    morecomment=[n]{(*}{*)},
    morecomment=[n]{<!--}{-->},
    commentstyle=\color{colComments},
    abovecaptionskip = 0.5ex,
    belowcaptionskip = \smallskipamount,
    aboveskip=2ex,
    belowskip=0.5ex,
    inputpath=SourceCode,
}

% Styles for Structured Text SCL / ST
\lstdefinelanguage{SCL}{
    numbersep = 15pt,
    identifierstyle=\color{colIdentifier},
    morecomment=[l]{//},
    morecomment=[s]{/*}{*/},
    morecomment=[n]{(*}{*)},
    commentstyle=\color{colComments},
    morestring=[b]{"},
    morestring=[b]{'},
    %stringstyle=\color{colString},
    stringstyle=\color{magenta},
    keywordstyle=\color{colKeys},
    keywordstyle=[2]\color{hellblau},
    keywordstyle=[3]\color{colName},
    morekeywords={CASE, END_CASE, OF, IF, ELSIF, ELSE, THEN, END_IF, LABEL, END_LABEL, GOTO, BEGIN,
                  FOR, TO, DO, BY, END_FOR, WHILE, END_WHILE, REPEAT, UNTIL, END_REPEAT, CONTINUE, EXIT, RETURN,
                  VAR_INPUT, VAR_IN_OUT, VAR_OUTPUT, VAR, RETAIN, VAR_TEMP, END_VAR, FUNCTION_BLOCK, END_FUNCTION_BLOCK, FUNCTION, END_FUNCTION, void, Void, VOID, DATA_BLOCK, END_DATA_BLOCK, TYPE, END_TYPE, STRUCT, Struct, END_STRUCT },
    keywords=[2]{AND, OR, XOR, NOT, MOD, true, false, },
    % function call
    keywords=[3]{FC_no, FB_no, DB_no, IN_a, IN_b, SHR, SHL, LEN, FIND, LEFT, RIGHT, MID, DELETE, INSERT, REPLACE, DWORD_TO_USINT, USINT_TO_BYTE, Chars_TO_Strg, STRG_VAL, VAL_STR, CONCAT, DINT_TO_DWORD, },
    literate=%
        {0}{{{\color{red!20!violet}0}}}1
        {1}{{{\color{red!20!violet}1}}}1
        {2}{{{\color{red!20!violet}2}}}1
        {3}{{{\color{red!20!violet}3}}}1
        {4}{{{\color{red!20!violet}4}}}1
        {5}{{{\color{red!20!violet}5}}}1
        {6}{{{\color{red!20!violet}6}}}1
        {7}{{{\color{red!20!violet}7}}}1
        {8}{{{\color{red!20!violet}8}}}1
        {9}{{{\color{red!20!violet}9}}}1
        {<}{{{\color{hellblau}<}}}1
        {>}{{{\color{hellblau}>}}}1
        {=}{{{\color{hellblau}=}}}1
        {<>}{{{\color{hellblau}<>}}}2
        {>=}{{{\color{hellblau}>=}}}2
        {<=}{{{\color{hellblau}<=}}}2
        {:=}{{{\color{black}:=}}}2
        {~}{{\textasciitilde}}1
        {\$}{{\textcolor{blue}{\$}}}1
}

% Styles for HTML
\lstdefinelanguage{HTML5}{
    numbersep = 15pt,
    identifierstyle=\color{colIdentifier},
    morecomment=[l]{//},
    morecomment=[s]{<!--}{-->},
    morecomment=[n]{<!--}{-->},
    commentstyle=\color{colComments},
    morestring=[b]{'},
    morestring=[b]{"},
    %stringstyle=\color{colString},
    stringstyle=\color{magenta},
    emph={<, >, \/},
    emphstyle=\color{colKeys},
    emph={[2]<,>,\/},emphstyle={[2]\color{blue}},
    %otherkeywords={<, >, \/}, % http://texblog.org/tag/otherkeywords/
    %morekeywords={\<, \>, \/},
    keywordstyle=\color{colKeys},
    keywords=[2]{html, head, title, meta, link, script, body, div, span, p, a, br, hr, h, h1, h2, h3, h4, h5, h6, ul, li, img, iframe, table, thead, tbody, tfoot, tr, th, td, form, fieldset, legend, label, input, select, option,
    },
    keywordstyle=[2]\color{colKeys},
    %moredelim=*[s][\color{colKeys}]{<}{>},
    keywords=[3]{http, equiv, content, rel, href, src, type, class, id, name, alt, action, method, value, for, checked, onabort, onblur, onchange, onclick, ondblclick, onerror, onfocus, onkeydown, onkeypress, onkeyup, onload, onmousedown, onmousemove, onmouseout, onmouseover, onmouseup, onreset, onselect, onsubmit, onunload,
    },
    keywordstyle=[3]\color{red},
    literate=%
        {0}{{{\color{red!20!violet}0}}}1
        {1}{{{\color{red!20!violet}1}}}1
        {2}{{{\color{red!20!violet}2}}}1
        {3}{{{\color{red!20!violet}3}}}1
        {4}{{{\color{red!20!violet}4}}}1
        {5}{{{\color{red!20!violet}5}}}1
        {6}{{{\color{red!20!violet}6}}}1
        {7}{{{\color{red!20!violet}7}}}1
        {8}{{{\color{red!20!violet}8}}}1
        {9}{{{\color{red!20!violet}9}}}1
        %{<}{{{\color{hellblau}<}}}1
        %{>}{{{\color{hellblau}>}}}1
}

% Styles for CSS
\lstdefinelanguage{CSS}{
    numbersep = 15pt,
    %alsodigit = {-}% Hier: Seite 40 Doku Version 1.0 (1.4.02)
    %alsoletter= {#}
    identifierstyle=\color{colIdentifier},
    morecomment=[l]{//},
    morecomment=[s]{/*}{*/},
    morecomment=[n]{/*}{*/},
    commentstyle=\color{colComments},
    morestring=[b]{'},
    morestring=[b]{"},
    %stringstyle=\color{colString},
    stringstyle=\color{magenta},
    %KEYWORDS
    %otherkeywords={<, >, \/}, % http://texblog.org/tag/otherkeywords/
    morekeywords={html, head, title, body, div, span, p, a, br, h, ul, li, img, iframe, table, thead, tbody, tfoot, tr, th, td, form, fieldset, legend, label, input, select, option, even, odd,% h1, h2, h3, h4, h5, h6,
    },
    keywordstyle=\color{cssKey},
    % CSS Design Selector
    keywords=[2]{margin, padding, position, z, index, float, clear, height, width, top, bottom, left, right, overflow, display, visibility, min, empty, cells, border, radius, vertical, align, text, decoration, color, background, list, style, image, spacing, collapse, filter, opacity,
    },
    keywordstyle=[2]\color{css},
    %
    keywords=[3]{table, layout, font, family, size, weight, border, style,
    },
    keywordstyle=[3]\color{css},
    % CSS PSEUDO selector
    keywords=[4]{root, empty, first, last, nth, only, child, of, type, link, visited, hover, active, focus, target, disabled, enabled, checked, valid, invalid, lang, not,
    },
    keywordstyle=[4]\color{cssPseudo},
    % CSS Class selector
    keywords=[6]{hide, leftfloat, clearfloat, ind_img, ind_txt, load_pic,},
    keywordstyle=[6]\color{cssClass},
    % CSS ID's selector
    keywords=[9]{HeaderContainer, headertitle, breadcrumbs, content, grafik_sprite, load_div, },
    keywordstyle=[9]\color{cssId},
        literate=%
            {0}{{{\color{black}0}}}1
            {1}{{{\color{black}1}}}1
            {2}{{{\color{black}2}}}1
            {3}{{{\color{black}3}}}1
            {4}{{{\color{black}4}}}1
            {5}{{{\color{black}5}}}1
            {6}{{{\color{black}6}}}1
            {7}{{{\color{black}7}}}1
            {8}{{{\color{black}8}}}1
            {9}{{{\color{black}9}}}1
}

% Styles for JavaScript
\lstdefinelanguage{JavaScript}{
    numbersep = 15pt,
    identifierstyle=\color{colIdentifier},
    morecomment=[l]{//},
    morecomment=[s]{/*}{*/},
    morecomment=[n]{/*}{*/},
    commentstyle=\color{colComments},
    morestring=[b]{'},
    morestring=[b]{"},
    %stringstyle=\color{colString},
    stringstyle=\color{magenta},
    %otherkeywords={<, >, \/}, % http://texblog.org/tag/otherkeywords/
    %morekeywords={\<, \>, \/},
    keywordstyle=\color{colKeys},
    keywords=[2]{var, in, if, else, switch, case, break, continue, default, while, do, for, function, return, false, true,
    },
    keywordstyle=[2]\color{colKeys},
    keywords=[3]{
    },
    keywordstyle=[3]\color{red},
    literate=%
        {0}{{{\color{red!20!violet}0}}}1
        {1}{{{\color{red!20!violet}1}}}1
        {2}{{{\color{red!20!violet}2}}}1
        {3}{{{\color{red!20!violet}3}}}1
        {4}{{{\color{red!20!violet}4}}}1
        {5}{{{\color{red!20!violet}5}}}1
        {6}{{{\color{red!20!violet}6}}}1
        {7}{{{\color{red!20!violet}7}}}1
        {8}{{{\color{red!20!violet}8}}}1
        {9}{{{\color{red!20!violet}9}}}1
        %{<}{{{\color{hellblau}<}}}1
        %{>}{{{\color{hellblau}>}}}1
}

\definecolor{colIdentifierCS}{rgb}{0,0,0}
\definecolor{colCommentsCS}{rgb}{0,0.501,0}
\definecolor{colNameCS}{rgb}{0.168,0.568,0.686}
\definecolor{colStringCS}{rgb}{0.639,0.082,0.082}

% Styles for C#
\lstdefinelanguage{CSharp}{
    language=[Sharp]C,
    language=csh,
    numbersep = 15pt,
    identifierstyle=\color{colIdentifierCS},
    commentstyle=\color{colCommentsCS},
    morecomment=[l]{//}, %use comment-line-style!
    morecomment=[s]{/*}{*/}, %for multiline comments
    morecomment=[n]{/*}{*/}, %for multiline comments
    stringstyle=\color{colStringCS},
    morestring=[b]{"},
    morestring=[s]{@"}{"},
    morestring=[b]{'},
    keywordstyle=\color{colKeys},
    keywordstyle=[2]\color{colNameCS},
    morekeywords={  abstract, event, new, struct,
                    as, explicit, null, switch,
                    base, extern, object, this,
                    bool, false, operator, throw,
                    break, finally, out, true,
                    byte, fixed, override, try,
                    case, float, params, typeof,
                    catch, for, private, uint,
                    char, foreach, protected, ulong,
                    checked, goto, public, unchecked,
                    class, if, readonly, unsafe,
                    const, implicit, ref, ushort,
                    continue, in, return, using,
                    decimal, int, sbyte, virtual,
                    default, interface, sealed, volatile,
                    delegate, internal, short, void,
                    do, is, sizeof, while,
                    double, lock, stackalloc,
                    else, long, static,
                    enum, namespace, string},
    keywords=[2]{FolderBrowserDialog, DialogResult, },
}
