% Anpassung des Seitenlayouts --------------------------------------------------
%   siehe Seitenstil.tex
% ------------------------------------------------------------------------------
\usepackage[
    automark, % Kapitelangaben in Kopfzeile automatisch erstellen
    headsepline, % Trennlinie unter Kopfzeile
    footsepline,
    ilines % Trennlinie linksb\"undig ausrichten
]{scrpage2}

% Anpassung an Landessprache ---------------------------------------------------
\usepackage{ngerman}
\usepackage[ngerman]{babel}

% Umlaute ----------------------------------------------------------------------
%   Umlaute/Sonderzeichen wie \"a\"u\"o\ss  direkt im Quelltext verwenden (CodePage).
%   Erlaubt automatische Trennung von Worten mit Umlauten.
% ------------------------------------------------------------------------------

%\usepackage[ansinew]{inputenc}
\usepackage[utf8, ansinew, latin1]{inputenc}
\usepackage[T5,T1]{fontenc}
\usepackage{textcomp} % Euro-Zeichen etc.

\usepackage[babel,german=quotes]{csquotes}
\usepackage{pifont}

% Schrift ----------------------------------------------------------------------
\usepackage{lmodern} % bessere Fonts
\usepackage{relsize} % Schriftgr\"o\ss e relativ festlegen
\usepackage{microtype}

\usepackage[
	labelfont={normalfont,bf},
	format=hang,
	justification=RaggedRight,
	singlelinecheck=false,
	hypcap=false]
{caption}


% Grafiken ---------------------------------------------------------------------
% Einbinden von JPG-Grafiken erm\"oglichen
\usepackage[dvips,final]{graphicx}
% hier liegen die Bilder des Dokuments
\graphicspath{{Grafiken/}}
% Wrapping text around figures
\usepackage{wrapfig}
% mehrere Bilder nebeneinander
\usepackage{subfigure}
%\usepackage[subfigure]{tocloft}
%\usepackage{subfig}

% zum Umflie\ss en von Bildern
\usepackage{floatflt}
\usepackage[section]{placeins} % Bilder werden nicht mehr in die n\"achste Section geschoben sondern steh in der zugeh\"origen Section in der sie aufgerufen werden

\usepackage{verbatim}

% make a Directorie Three
\usepackage{dirtree}

% PSTricks - PAPs, Directorie Tree
\usepackage[inactive]{pst-pdf}
\usepackage{pstricks,pst-node}
% Text Boxen, Programm Ablauf Plaene
\usepackage{tikz}
\usetikzlibrary{
	positioning,
	arrows,
	calc,
	shapes.multipart,
	shapes
} 


\newcommand{\papYes}{\ding{51}} % hacken
\newcommand{\papNo}{\ding{55}}  % X Zeichen

% Befehle aus AMSTeX f\"ur mathematische Symbole z.B. \boldsymbol \mathbb --------
\usepackage{amsmath,amsfonts, amssymb}


% Einfache Definition der Zeilenabst\"ande und Seitenr\"ander etc. -----------------
\usepackage{setspace}
\usepackage{geometry}

% f\"ur Index-Ausgabe mit \printindex --------------------------------------------
%\usepackage{makeidx}
\usepackage[makeindex]{imakeidx} 
\makeindex[options={-s indexstill}, intoc = false]
\indexsetup{firstpagestyle=scrheadings}

\makeatletter
% we don't want a page break before a subitem
\renewcommand\subitem{\@idxitem\nobreak\hspace*{20\p@}}
\makeatother

% Symbolverzeichnis ------------------------------------------------------------
%   Symbolverzeichnisse bequem erstellen. Beruht auf MakeIndex:
%     makeindex.exe %Name%.nlo -s nomencl.ist -o %Name%.nls
%   erzeugt dann das Verzeichnis. Dieser Befehl kann z.B. im TeXnicCenter
%   als Postprozessor eingetragen werden, damit er nicht st\"andig manuell
%   ausgef\"uhrt werden muss.
%   Die Definitionen sind ausgegliedert in die Datei "Glossar.tex".
% ------------------------------------------------------------------------------
%\usepackage[intoc]{nomencl}
\usepackage{nomencl}
% Befehl umbenennen in abk
\let\abk\nomenclature
\setlength{\nomlabelwidth}{0.26\textwidth}
\renewcommand{\nomlabel}[1]{#1 \dotfill}
\setlength{\nomitemsep}{-\parsep}
\makenomenclature

% zum Einbinden von Programmcode -----------------------------------------------
\usepackage{listings}

%Farben ------------------------------------------------------------------------
\usepackage{xcolor}

% URL verlinken, lange URLs umbrechen etc. -------------------------------------
\usepackage{url}

% wichtig f\"ur korrekte Zitierweise ---------------------------------------------
%\usepackage[square, numbers, sectionbib]{natbib}
\usepackage[square, comma, numbers, sort&compress]{natbib}

%\usepackage[
%	backend=bibtex,
%	style=verbose-inote,
%	isbn=false,
%	url=false,
%	block=space,
%	pagetracker=true,
%	sorting=nty
%]{biblatex}
%\usepackage[backend=biber]{biblatex}


% fortlaufendes Durchnummerieren der Fu\ss noten ----------------------------------
\usepackage{chngcntr}

% f\"ur lange Tabellen -----------------------------------------------------------
\usepackage{longtable}
\usepackage{tablefootnote}
\usepackage{footnote}
\usepackage{diagbox}
\usepackage{array}
\usepackage{ragged2e}
\usepackage{lscape}

% Spaltendefinition rechtsb\"undig mit definierter Breite ------------------------
\newcolumntype{w}[1]{>{\raggedleft\hspace{0pt}}p{#1}}

% Formatierung von Listen \"andern -----------------------------------------------
\usepackage{paralist}

% bei der Definition eigener Befehle ben\"otigt
\usepackage{ifthen}

% definiert u.a. die Befehle \todo und \listoftodos
\usepackage{todonotes}

% sorgt daf\"ur, dass Leerzeichen hinter parameterlosen Makros nicht als Makroendezeichen interpretiert werden
\usepackage{xspace}