\chapter{Vorlagen Chapter \& To Do's}
\label{cha:test_chap}

\section{Test Section}
\label{sec:test_sec}

\subsection{reserve Section}
\label{subsec:guideline_res}

\subsubsection{reserve Section}
\label{subsubsec:guideline_res}

\url{https://developers.google.com/speed/pagespeed/}


Test abschnitt\footnote{nach \citet[5. Code minimieren \& optimieren]{smart_webentwicklung}\label{ftn:smart-webentwicklung}}

zweiter Testabschnitt\footref{ftn:smart-webentwicklung}}

\section[Titel des Abschnitts]{Titel des Abschnitts\footnote{Fasse dich kurz!}}
\subsection{Test SubSection}
\label{subsec:test_subsec}

Test zum einbinden einer animierten GIF-Grafik
\begin{figure}[htp]
\centering
\caption[Ladeindikator]{Ladeindikator}
\label{fig:ladeindikator}
\end{figure} 

\newpage

\section{Quellcodes}
\label{sec:listing}

\subsection{Quellcodes mit Listing Paket}
\label{subsec:listing_quellcode}

\lstset{language=CSS,basicstyle=\footnotesize}
\begin{lstlisting}[caption={[QC im Verzeichniss \newline{} \Quelle{\citet{css_sprites_animation}}]\CSS QC unter listing\protect\footnotemark}, label=lst:QC_test]
#load_div{
position: absolute;
top: 136px;		left: 118px;
bottom: 0;		right: 17px;
z-index: 1000;	background-color: #ffffff; 
filter: alpha(opacity=60);		opacity: 0.6;
}
\end{lstlisting}
\footnotetext{\Quelle{\citet{css_sprites_animation}}}

\newpage

\section{Programmablaufpl\"ane (PAP)}
\label{sec:pap}

\subsection{PAP mit TIKZ}
\label{subsec:pap_tikz}

\begin{figure}[htp]
\centering
\begin{tikzpicture}[node distance = 3.0em, auto]
\node [papSE](Start1)
		{Start};
\node [papDescision, below of = Start1, yshift= -3.5em](Desc1)
		{word word word};
\node [papInput, below of = Desc1, yshift= -3.5em](input)
		{User Input};
\node [papData, below of = input, yshift= -1.0em](Trap1)
		{Data read};
\node [papFunction,  below of = Trap1] (Func1)
		{\nodepart{two}\shortstack{Predefined\\Process}};
\node [papFunction,  right of = Trap1, xshift=7.5em] (Func2) 
		{\nodepart{two}Function\\2, second};
\node [papInstruction,  below of = Func2](Func3)
		{Operation};
\node [papSE, below of = Func1, yshift= -1em](End)
		{Ende};

\coordinate [below of = Func1, yshift= 5mm] (join1);


\path [papLine] (Start1) -- (Desc1);
\path [papLine] (Desc1) -- node [at start] {\papYes} (input);
\path [papArrow] (input) -- (Trap1);
\path [papLine] (Func3) |- (join1);
\path [papLine] (Func1) -- (End);
\path [papLine] (Desc1) -| node [at start] {\papNo} (Func2);
\path [papLine] (Func2) -- (Func3);
\path [papLine] (Trap1) -- (Func1);

\end{tikzpicture}
\caption[PAP Muster]{PAP Muster}
\label{pap:pap_muster}
\end{figure}

\begin{center}
\begin{minipage}{\linewidth}
\centering
\begin{tikzpicture}[auto]
\node [papSE](ExpStart)
		{Start};
\node [papData, below of=ExpStart, yshift=-2.0em](ExpData)
		{Datenexport aus TIA-Portal\\ - Datenbausteine\\ - Benutzerdefinierte Typen\\ - Symboltabellen};
\node [papSE, below of = ExpData, yshift= -2.0em](ExpEnde)
		{Ende};

\path [papArrow] (ExpStart) -- (ExpData);
\path [papArrow] (ExpData) -- (ExpEnde);

\node [papSE, right of = ExpStart, xshift= 17.5em](ComStart)
		{Start};
\node [papInput, below of=ComStart, yshift=-2.5em](ComSystem)
		{System w\"ahlen: \\$\bigotimes$ S7-1200\\$\boldsymbol{\bigcirc}$ S7-1500};
\node [papInput, below of=ComSystem, yshift=-2.5em](ComDataFiles)
		{Auswahl der Datendateien\\ - S7 Variablen};
\node [papData, below of=ComDataFiles, yshift=-1.0em](ComData)
		{Datenimport aus TIA-Portal\\ DB, UDT, Vaiablentabellen};
\node [papFunction, below of=ComData, yshift= -1.2em](ComDataImport)
		{\nodepart{two}Import der Variablen mit Typangabe\\ Erstellen eines Protokolls};
\node [papInput, below of=ComDataImport, yshift=-2.0em](ComHtmlFiles)
		{Auswahl der Datendateien\\ - Webinterface};
\node [papData, below of=ComHtmlFiles, yshift=-1.0em](ComHtml)
		{Datenimport der Webinterface Dateien\\ HTML, HTM, JS};
\node [papFunction, below of=ComHtml, yshift= -1.2em](ComHtmlImport)
		{\nodepart{two}Kompilieren der Webinterface Dateien\\ Erstellen eines Protokolls};
\node [papFunction, below of=ComHtmlImport, yshift= -2.0em](ComGen)
		{\nodepart{two}Generierung der Scripte\\ - SCL Quellen\\ - JavaScript, JSON};
\node [papInput, below of=ComGen, yshift=-2.5em](ComTargetFolder)
		{Auswahl des Zielordners};
\node [papData, below of=ComTargetFolder, yshift=-0.8em](ComSave)
		{Datenexport der Dateien\\ HTML, HTM, JS, JSON, SCL, LOG};
\node [papSE, below of = ComSave, yshift= -1.5em](ComEnde)
		{Ende};

\path [papArrow] (ComStart) -- (ComSystem);
\path [papArrow] (ComSystem) -- (ComDataFiles);
\path [papArrow] (ComDataFiles) -- (ComData);
\path [papArrow] (ComData) -- (ComDataImport);
\path [papArrow] (ComDataImport) -- (ComHtmlFiles);
\path [papArrow] (ComHtmlFiles) -- (ComHtml);
\path [papArrow] (ComHtml) -- (ComHtmlImport);
\path [papArrow] (ComHtmlImport) -- (ComGen);
\path [papArrow] (ComGen) -- (ComTargetFolder);
\path [papArrow] (ComTargetFolder) -- (ComSave);
\path [papArrow] (ComSave) -- (ComEnde);

\node [papSE, left of = ComEnde, xshift= -17.5em](ImpEnd)
		{Ende};
\node [papData, above of=ImpEnd, yshift=2.0em](ImpData)
		{Datenimport in TIA-Portal\\ - SCL-Quellen\\ - Webinterface};
\node [papSE, above of = ImpData, yshift= 2.0em](ImpStart)
		{Start};

\path [papArrow] (ImpStart) -- (ImpData);
\path [papArrow] (ImpData) -- (ImpEnd);

\coordinate [right of = ExpEnde, xshift= 7.0em] (join1);
\path [papLine] (ExpEnde) -- (join1);
\path [papArrow] (join1) |- (ComStart);

\coordinate [right of = ImpEnd, xshift= 7.0em] (join2);
\path [papLine] (ComEnde) -- (join2);
\path [papArrow] (join2) |- (ImpStart);
\end{tikzpicture}
\captionof{figure}[WebOptimierungsTool - Ablauf]{WebOptimierungsTool - Ablauf}
\label{pap:WebGen_konzept_test}
\end{minipage}
\end{center}




