% !TeX root = ../Bachelorarbeit.tex

\chapter{Fazit und Ausblick}
\label{cha:fazit}

\emph{Die Ergebnisse wurden bisher nur dargestellt und erklärt und nicht bewertet und analysiert, dass soll hier geschehen unter "erreichte Ziele". Das Fazit soll Punkte zu Anwendbarkeit der Technologien, Entwicklung/Nachbau eines Spieleklassikers und Problempunkte, aber auch positives zu Support und Inbetriebnahme enthalten. Der Ausblick soll kommende Technologien und weitere Entwicklungspunkte für das Spiel beleuchten.}

\section{Fazit}
\label{sec:Schlussfolgerung}\index{Fazit}

\subsection{erreichte Ziele}
\label{subsec:erreichte_ziele}

\section{Ausblick}
\label{sec:Ausblick}\index{Ausblick}

Die Entwicklung dieses Ablegers von Archon ist keinesfalls abgeschlossen. Es gibt viele offene Features und Themen für weitere Abschlussarbeiten.

\subsubsection{Generelle Weiterentwicklung}

Das Spiel an sich ist noch nicht komplett reproduziert worden.

\subsubsection{KI-Modus / Demo-Modus}

Diese Realisierung hat bisher nur menschliche Spieler realisiert, jedoch erlaubt die Architektur genauso den Ersatz der View-Komponente durch einen Computer-Spieler. Eine spannende Arbeit kann daher die Entwicklung einer künstlichen Intelligenz (KI) für den Demo-Modus \bzw den Modus Spieler-vs-Computer sein.

\subsubsection{Score-Boards}

Es können eine Datenbank und ein Authentifizierungsservice angebunden werden und damit Spieler registriert werden. Das würde die Implementierung einer Rangliste (Score-Board) oder eines Systems von Spielerfolgen (Achievements) ermöglichen.

\subsubsection{Verschiedenes}