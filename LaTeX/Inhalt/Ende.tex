% !TeX root = ../Bachelorarbeit.tex

\chapter{Fazit und Ausblick}
\label{cha:fazit}

\emph{Die Ergebnisse wurden bisher nur dargestellt und erklärt und nicht bewertet und analysiert, dass soll hier geschehen unter "erreichte Ziele". Das Fazit soll Punkte zu Anwendbarkeit der Technologien, Entwicklung/Nachbau eines Spieleklassikers und Problempunkte, aber auch positives zu Support und Inbetriebnahme enthalten. Der Ausblick soll kommende Technologien und weitere Entwicklungspunkte für das Spiel beleuchten.}

\section{Fazit}
\label{sec:Schlussfolgerung}\index{Fazit}

\subsection{erreichte Ziele}
\label{subsec:erreichte_ziele}

\section{Ausblick}
\label{sec:Ausblick}\index{Ausblick}

Die Entwicklung dieses Ablegers von Archon ist mit dieser Arbeit keinesfalls abgeschlossen. Es gibt viele offene Features und Themen für weitere Abschlussarbeiten, die im Folgenden kurz angerissen werden.

\subsubsection{Generelle Weiterentwicklung}

Das Spiel an sich ist noch nicht komplett reproduziert worden. Es können mehr Sprites implementiert und eine bessere visuelle Erfahrung entwickelt werden.\\
Weiterhin kann die HTML-Audio-API benutzt werden um Soundeffekte einzufügen.\\
Die Kampf-Komponente ist noch nicht wirklich fertiggestellt und auch die Zauber sind noch nicht fertiggestellt. Daher ist hier noch Bedarf für weitere Entwicklungsarbeiten.

\subsubsection{KI-Modus / Demo-Modus}

Diese Realisierung hat bisher nur menschliche Spieler realisiert, jedoch erlaubt die Architektur genauso den Ersatz der View-Komponente durch einen Computer-Spieler. Eine spannende Arbeit kann daher die Entwicklung einer künstlichen Intelligenz (KI) für den Demo-Modus \bzw den Modus Spieler-vs-Computer sein.

\subsubsection{Score-Boards}

Es können eine Datenbank und ein Authentifizierungsservice angebunden werden und damit Spieler registriert werden. Das würde die Implementierung einer Rangliste (Score-Board) oder eines Systems von Spielerfolgen (Achievements) ermöglichen.

\subsubsection{Verschiedenes}

Das Spiel ist gemacht für die Benutzung mit einem GamePad\footnote{Ein GamePad ist ein Spielecontroller mit Joystick und diverses generellen Tasten.}. Die HTML-GamePad-API ist allerdings noch in einem sehr frühen Stadium der Standardisierung, sodass zunächst eine Evaluierung stattfinden muss, ob sie ausreichend benutzbar ist, oder ob andere Bibliotheken diese Funktionalität hergeben.

Außerdem kann der Inputcontroller um die Touch-Events für die Benutzung auf mobilen Endgeräten erweitert werden.