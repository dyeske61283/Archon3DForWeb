% !TeX root = ../Bachelorarbeit.tex

\chapter{Fazit und Ausblick}
\label{cha:fazit}

\section{Fazit}
\label{sec:Schlussfolgerung}\index{Fazit}

Auch ein Spiel, dass aus den Anfängen der digitalen Spielwelt stammt kann einen erheblichen Entwicklungsaufwand erzeugen. Diese Neuauflage von Archon war da keine Ausnahme. Archon ist als Spiel sehr vielseitig, was seine Reize und Beliebtheit und auch die Reize dieser Realisierung ausmacht. Die verschiedenen Technologien, die in diese Realisierung einfließen mussten um eine Realisierung überhaupt möglich zu machen zeigen dies sehr deutlich.\\
Auch hat diese Abschlussarbeit gezeigt, dass ein strukturiertes Vorgehen, sorgfältige Vorentwicklungen und das Anfertigen von Prototypen die eigentliche Entwicklung stark vereinfachen und weniger fehleranfällig machen. Natürlich kosten diese Maßnahmen auch Zeit, sodass leider nicht alle nötiges Features implementiert werden konnten. Aber es ist eine gut wartbare und erweiterbare Webapplikation entstanden. Die Applikation konnte nicht alle Ziele erfüllen, die dieser Abschlussarbeit gesetzt waren, bietet aber eine solide Grundlage für Folgearbeiten.\\
Das Anfertigen der Realisierung von Archon mit 3D- und Webtechnologien hat breite Kenntnisse in den Gebieten der modernen Webentwicklung vermittelt und gezeigt, dass das Web eine universelle Plattform darstellt, die man nur mit Ideen füllen muss.\\
Insgesamt stellt diese Arbeit daher eine gute Sammlung der benutzten Technologien dar. Auch ist trotzdem die Verschmelzung von State-of-the-Art Webentwicklung mit einem sehr bekannten Spiel der 1980er Jahre gut gelungen.


\section{Ausblick}
\label{sec:Ausblick}\index{Ausblick}

Die Entwicklung dieses Ablegers von Archon ist mit dieser Arbeit daher noch nicht abgeschlossen. Es gibt viele offene Features und Themen für weitere Abschlussarbeiten, die im Folgenden kurz angerissen werden.

\subsubsection{Generelle Weiterentwicklung}

Das Spiel an sich ist noch nicht komplett reproduziert worden. Es können mehr Sprites implementiert und eine bessere visuelle Erfahrung entwickelt werden.\\
Weiterhin kann die HTML-Audio-API benutzt werden um Soundeffekte einzufügen.\\
Die Kampf-Komponente ist noch nicht wirklich fertiggestellt und auch die Zauber sind noch nicht fertiggestellt. Daher ist hier noch Bedarf für weitere Entwicklungsarbeiten.

\subsubsection{KI-Modus / Demo-Modus}

Diese Realisierung hat bisher nur menschliche Spieler realisiert, jedoch erlaubt die Architektur genauso den Ersatz der View-Komponente durch einen Computer-Spieler. Eine spannende Arbeit kann daher die Entwicklung einer künstlichen Intelligenz (KI) für den Demo-Modus \bzw den Modus Spieler-vs-Computer sein.

\subsubsection{Score-Boards}

Es können eine Datenbank und ein Authentifizierungsservice angebunden werden und damit Spieler registriert werden. Das würde die Implementierung einer Rangliste (Score-Board) oder eines Systems von Spielerfolgen (Achievements) ermöglichen.

\subsubsection{Verschiedenes}

Das Spiel ist gemacht für die Benutzung mit einem GamePad\footnote{Ein GamePad ist ein Spielecontroller mit Joystick und diverses generellen Tasten.}. Die HTML-GamePad-API ist allerdings noch in einem sehr frühen Stadium der Standardisierung, sodass zunächst eine Evaluierung stattfinden muss, ob sie ausreichend benutzbar ist, oder ob andere Bibliotheken diese Funktionalität hergeben.\\
Außerdem kann der Inputcontroller um die Touch-Events für die Benutzung auf mobilen Endgeräten erweitert werden.

Desweiteren ist bisher keine Verschlüsselung des Datenverkehrs und auch kein Zugriff auf die Webseite via HTTPS umgesetzt.

Als Abschluss ist eine Parallelisierung der Architektur ein offener Punkt. Dies erlaubt es mehrere Spiele gleichzeitig auf einem Server spielen zu lassen, dafür müssten auch Leistungs- und Lasttests durchgeführt werden, um leistungsgemäß viele Spiele laufen zu lassen.