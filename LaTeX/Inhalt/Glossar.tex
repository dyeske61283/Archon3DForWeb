\abk{o.V.}{ohne Verfasser (bei Literaturverweisen)}
\abk{Abb.}{Abbildung}
\abk{API}{Programmschnittstelle nach aussen (engl. f{\"u}r Application Programming Interface)}
\abk{SDK}{\sa IDE (engl f{\"u}r Software Development Kit)}
\abk{IDE}{integrierte Entwicklungsumgebung (engl. f{\"u}r Integrated Development Environment)}

\abk{W3C}{Organisation zur Standardisierung von Webtechnologien (engl. f{\"u}r World Wide Web Consortium)}
\abk{WWW}{Internet (engl. f{\"u}r World Wide Web)}
\abk{DOM}{Dokumentstruktur der Webseite (engl. f{\"u}r Document Object Model)}
\abk{HTML}{Hypertext Auszeichnungssprache, HTML-Dateien sind die Grundlage des World Wide Web und werden von einem Webbrowser dargestellt (engl. f{\"u}r Hypertext Markup Language)}
\abk{HTTP}{Hypertext-\"Ubertragungsprotokoll, Protokoll zur \"Ubertragung von Daten \"uber ein Netzwerk  (engl. f{\"u}r Hypertext Transfer Protocol)}
\abk{HTTPS}{sicheres Hypertext-\"Ubertragungsprotokoll \sa HTTP (engl. f\"ur Hypertext Transfer Protocol Secure)}
\abk{Webseite}{\sa HTML-Datei}
\abk{CSS}{gestufte Gestaltungsb\"ogen, legt die Darstellung des HTML Quellcodes im Browser fest (engl. f{\"u}r Cascading Style Sheets)}
\abk{JavaScript}{Skriptsprache, urspr\"unglich f\"ur dynamisches HTML in Webbrowsern entwickelt}
\abk{JSON}{kompaktes Datenformat zum Datenaustausch mit \zB Webservern (engl. f{\"u}r JavaScript Object Notation)}
\abk{Webbrowser}{auch kurz Browser (engl. to browse) steht f{\"u}r durchst{\"o}bern, abgrassen, durchsuchen - Software zum Darstellen von Daten, haupts{\"a}chlich Webseiten und deren Inhalt, k\"onnen zu diesem Zweck mit Webservern kommunizieren}
\abk{Webserver}{Ein Webserver speichert Webseiten und stellt diese zur Verf\"ugung. Der Webserver ist eine Software, die Dokumente mit Hilfe standardisierter \"Ubertragungsprotokolle (HTTP, HTTPS) an einen Webbrowser \"ubertr\"agt.}
\abk{GUI}{Grafisches Benutzer Interface (engl. f{\"u}r Graphical User Interface)}
\abk{Interface}{Schnittstelle}
\abk{JPEG}{komprimierte Grafikdatei, auch JPG (engl. f\"ur Joint Photographic Expert Group)}
\abk{GIF}{Grafikaustausch Format, Animationsf\"ahig (engl. f\"ur Graphics Interchange Format)}
\abk{PNG}{Grafikaustausch Format (engl. f\"ur Portable Network Graphics)}
\abk{SVG}{skalierbare Vektorgrafik (engl. f\"ur Scalable Vector Graphics)}

