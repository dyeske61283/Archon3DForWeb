\chapter{Einleitung}
\label{cha:Einleitung}

\section{Hinführung}
\label{sec:hinfuhrung}\index{Hinführung}

\emph{Die Hinführung soll den Leser an das Thema 3D-Technologien, Web-Entwicklung und Spieleklassiker, wie Archon heranführen. Hier sind Quellen und Recherchen zu ein paar Zahlen, wie Remakes, Umsätze zu Uralt-Spielen aktuell und Beliebtheit/Zahlen zum Webbrowser als Spieleplattform angebracht, um Wichtigkeit der Plattform und Technologien, aber auch Relevanz für Spieleklassiker zu zeigen.}

\begin{itemize}
	\item Verkaufte Konsolen
	\item Produzierte Spiele und Verkäufe
	\item Neuauflagen von Spielen und Konsolen
	\item Neuauflagen von Archon 
	\item Virtualisierung und Emulatoren um alte Spiele zum Anfassen zu bekommen
	\item Aber, nur auf richtigen PCs zu finden und meist nicht wirklich legal
	\item Web als freie, riesige Plattform mit großem Zugriffsbereich
	\item Web APIs, die in den letzten Jahren erschienen sind (kurze Recherche zu Arbeiten des W3C!)
	\item Schiere Vielfalt an JS-Frameworks zu Web APIs und Komfort bei der Entwicklung (npm packages um Zahlen zu bekommen)
	\item Beispiele zu 3D-Möglichkeiten im Web, Stichwort: THREE.JS-Demos
\end{itemize}

\section{Motivation / Problemstellung}
\label{sec:motivation}\index{Motivation}
\emph{ Diese Sektion soll vom vorherigen Stand der Forschung und Möglichkeiten auf Anwendbarkeit überleiten und zeigen, dass die evtl. surreal erscheinende Vermischung von Genres, Plattformen und Technologien anwendbar ist, und das sogar mit beschränktem Aufwand. Vereinbarkeit von aktuellen Technologien mit Spiele-Klassikern und Spiele-Klassiker als Anwendungsbeispiele dafür.}
Bisher sind oftmals nur Demonstrationen der einzelnen Technologien zu finden, oder kurze Starter-Kits bzw. Einstiegspunkte in die jeweilige Technologie, ...\\
Versionen, Ableger und Kopien des Spieleklassikers Archon gibt es zahlreich. Meist sind diese jedoch auf ein System beschränkt, oder tatsächlich nur mit einem Emulator ausführbar. Das Web leistet an dieser Stelle die perfekte Abhilfe: Es gibt unzählige Geräte, die heutzutage einen Webbrowser installiert haben, somit hat jeder Zugriff auf diesen Ableger, der ein webfähiges Gerät besitzt.

\section{Ziele dieser Arbeit}
\label{sec:ziele_dieser_arbeit}\index{Ziele}

\emph{Hier soll bloß kurz und prägnant die Fragestellung genannt werden und eine kurze Analyse der einzelnen Worte des Titels stattfinden. Ein Satz, wie "Die exakten Anforderungen für die Umsetzung dieser groben Ziele werden zu einem späteren Zeitpunkt herausgearbeitet", oder "... Anforderungen werden aus dem Aufbau der Arbeit heraus klar." muss hier hinein.}

\section{Aktueller Forschungsstand}
\label{sec:aktueller_forschungsstand}\index{Forschungsstand}

\emph{Hier sollen dann Extrema zum einen in Richtung 3D-Entwicklung, in Richtung Web-Entwicklung und aber auch bei der Spiele-Entwicklung kurz herausgearbeitet werden, um dem Leser einen kurzen Einblick in aktuelle Möglichkeiten zu geben, und zu zeigen, dass die Arbeit "State of the Art" ist.}

Fliegt vermutlich raus, oder kommt an andere Position.

\section{Zentrale Begriffe}
\label{sec:zentrale_begriffe}\index{Zentrale Begriffe}

\subsection{Was versteht man unter Webtechnologien?}
Webtechnologien betiteln die Sammlung aller nötigen Aspekte zum Erstellen einer Webanwendung.\\
Webanwendungen selbst bestehen aus einem Client-Server-Modell.
Typische Bestandteile des Clients sind: 
\begin{itemize}
	\item HTML zur Beschreibung des Inhalts
	\item CSS zur Beschreibung des Aussehens
	\item JavaScript zur Dynamisierung des Clients
\end{itemize}
Ein Webserver gibt auf HTTP-Anfragen die entsprechenden Inhalte und Medien an den Client heraus, welcher durch einen Webbrowser angezeigt wird. Außerdem können hier Daten des Clients verarbeitet, gespeichert und verteilt werden werden.
\subsection{Was versteht man unter 3D-Technologien?}
Da ein Bildschirm, beispielweise eines Computers, nur zweidimensional ist, muss durch andere Methodiken der Effekt einer dritten Dimension geschaffen werden.\\Objekte im 3D-Raum werden über ihre Eckpunkte aufgespannt und im Code somit in einem 3D-Raum mit normalem 3-Achsen-Koordinatensystem dargestellt. Anschließend folgt der Prozess des Renderns, bei dem zunächst aus den einzelnen Punkten der Objekte die Formen errechnet werden, indem die Eckpunkte zu Flächen verbunden werden. Anschließend erfolgt eine Ausrichtung aller Flächenpunkte am Pixel-Raster, sodass eine 3D-Projektion der 2D Pixel entsteht. Der nächste Schritt, Fragment-Bearbeitung, behandelt die Einfärbung der, im vorherigen Schritt gebildeten, "Fragmente" anhand von Licht und verwendeten Texturen. Der finale Schritt des Renderns wandelt die 3D-Projektion in ein 2D-Pixel-Bild, dass dann auf dem Bildschirm angezeigt wird. Außerdem werden hier Prüfungen für die Sichtbarkeit von Objekten unternommen, sodass nicht sichtbare Objekte, oder Teile von ihnen, auch nicht weiter verarbeitet werden.
\section{Aufbau}
\label{sec:aufbau}\index{Aufbau}

\emph{Hier wird dann auf den Aufbau des Haupt- und Schlussteils eingegangen. Also der erste Abschnitt beschäftigt sich mit Anforderungsanalyse beider Plattformen/Seiten. Der zweite Abschnitt dreht sich dann rund um Zusammenführung der Anforderungen und Tool-Auswahl, und somit darum einen Einstiegspunkt und eine Architektur für die weitere Entwicklung festzulegen. }

Im ersten Teil dieser Arbeit sollen die Anforderungen an eine Neuauflage von "Archon" erfasst werden und eine Beschreibung der Features und Spielmechaniken zur Umsetzung erstellt werden.
Anschließend werden die Anforderungen der technologischen Seite herausgearbeitet, sodass eine Liste an Funktionalitäten entsteht, die von den 3D- und Webtechnologien erfüllt werden muss.
Alle Anforderungen werden dann in eine Beschreibung zur Umsetzung und eine entsprechende Architektur ausgearbeitet.
Den Abschluss bildet eine Überprüfung der fertigen Neuauflage auf die Erfüllung der Anforderungen, sowie eine Präsentation und eine Reflexion des Ergebnisses.