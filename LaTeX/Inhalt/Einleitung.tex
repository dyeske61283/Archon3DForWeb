\chapter{Einleitung}
\label{cha:Einleitung}

\emph{Die Einleitung soll hier nur in ihrer Struktur erkl\"art werden, die anderen Unterteile f\"uhren an das Thema heran, erkl\"aren es und beschreiben den Aufbau der restlichen Arbeit}

\section{Hinführung}
\label{sec:hinfuhrung}\index{Hinführung}

\emph{Die Hinf\"uhrung soll den Leser an das Thema 3D-Technologien, Web-Entwicklung und Spieleklassiker, wie Archon heranführen. Hier sind Quellen und Recherchen zu ein paar Zahlen, wie Remakes, Umsätze zu Uralt-Spielen aktuell und Beliebtheit/Zahlen zum Webbrowser als Spieleplattform angebracht, um Wichtigkeit der Plattform und Technologien, aber auch Relevanz für Spieleklassiker zu zeigen.}

\section{Aktueller Forschungsstand}
\label{sec:aktueller_forschungsstand}\index{Forschungsstand}

\emph{Hier sollen dann Extrema zum einen in Richtung 3D-Entwicklung, in Richtung Web-Entwicklung und aber auch bei der Spiele-Entwicklung kurz herausgearbeitet werden, um dem Leser einen kurzen Einblick in aktuelle M\"oglichkeiten zu geben, und zu zeigen, dass die Arbeit "State of the Art" ist. }

\section{Motivation}
\label{sec:motivation}\index{Motivation}
\emph{ Diese Sektion soll vom vorherigen Stand der Forschung und Möglichkeiten auf Anwendbarkeit überleiten und zeigen, dass die evtl. surreal erscheinende Vermixung von Genres, Plattformen und Technologien anwendbar ist, und das sogar mit beschränktem Aufwand. }

\section{Ziele dieser Arbeit}
\label{sec:ziele_dieser_arbeit}\index{Ziele}

\emph{Hier soll bloß kurz und prägnant die Fragestellung genannt werden und eine kurze Analyse der einzelnen Worte des Titels stattfinden.}
\todo{Hier fehlt irgendwie direkt mal eine sinnvolle \"Uberleitung/Reihenfolge von Motivation zu Ziele zu zentrale Begriffe.}

\section{Zentrale Begriffe}
\label{sec:zentrale_begriffe}\index{Zentrale Begriffe}

\emph{?????}

\section{Aufbau}
\label{sec:aufbau}\index{Aufbau}

\emph{Hier wird dann auf den Aufbau des Haupt- und Schlussteils eingegangen. Also der erste Abschnitt beschäftigt sich mit Anforderungsanalyse beider Plattformen/Seiten. Der zweite Abschnitt dreht sich dann rund um Zusammenf\"uhrung der Anforderungen und Tool-Auswahl, und somit darum einen Einstiegspunkt und eine Architektur f\"ur die weitere Entwicklung festzulegen. }