\chapter*{Vorwort}
\label{cha:Vorwort}
\addcontentsline{toc}{chapter}{\nameref{cha:Vorwort}}

Im Jahr 2017 wurden mehr als 1,8 Milliarden browser-fähige Endgeräte weltweit verkauft. Der technologische Fortschritt in Hardware und Software erlauben es dieser großen Masse an Zugängen zum Internet eine große Vielfalt an Applikationen nahezu überall zu nutzen. Seien es Business-, oder Unterhaltungs-Applikationen, und auch Spiele jeglicher Art.\\
Die klaren Vorteile des Web als Plattform, auch für Spiele, sind die Offenheit, der einfache Zugang und die immer weiter steigende Unterstützung neuer Technologien.\\
Sogenannte Spiele-Engines, also Bibliotheken die sich um die meisten, generellen Problemstellungen eines Spiels und seiner Entwicklung kümmern, gibt es daher allerlei.
So sind heutzutage hardwarebeschleunigte 3D-Visualisierung und Echtzeitkommunikation im Webbrowser Stand der Technik.
In dieser Arbeit soll die Verschmelzung von State of the Art Webtechnologien mit einem Retro-Spiel erfolgen.\\
"Archon" genoss für Atari und C64 großen Erfolg, ähnlich vieler anderer Spiele für die Konsolen der 80er.
\todo{Anmerkung \& Recherche zu Erfolg von Archon belegen!}
Durch die strategischen und actionreichen Aspekte in einem Spiel hat es viel Abwechslung, auch aus Sicht eines potentiellen Entwicklers für einen Ableger des Spiels. Das Spiel dient daher als sehr gutes Exempel für die Möglichkeiten des Webs als Spieleplattform, aber auch der Möglichkeiten und Freiheiten des Webs für jede andere Art von Applikation.