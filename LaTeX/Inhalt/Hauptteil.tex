\chapter{Hauptteil}
\label{cha:Hauptteil}

\section{Einleitung}

\section{Analyse}
\label{sec:analyse}\index{Analyse}

The goal of every video game is to present the user(s) with a situation, accept their input, interpret those signals into actions, and calculate a new situation resulting from those acts. Games are constantly looping through these stages, over and over, until some end condition occurs (such as winning, losing, or exiting to go to bed). Not surprisingly, this pattern corresponds to how a game engine is programmed. The specifics depend on the game. \cite{https://developer.mozilla.org/en-us/docs/games/anatomy}

\emph{In diesem Abschnitt soll die Vorgehensweise bei der Analyse kurz aufgezeigt werden und der Inhalt der einzelnen Teile klar gemacht werden. Also, zum einen, dass hier eben eine Anforderungsanalyse stattfindet, und zum Anderen, dass diese Analyse zweigeteilt ist.}

\subsection{Analyse des klassischen Spiels}
\label{subsec:spiel_analyse}\index{Spiel Analyse}

\emph{Hier werden alle Aspekte des klassischen Spiels herausgearbeitet, gegliedert nach Teilen: Generelle Regeln, das Board, die Figuren, Abläufe und Wirkungen, mögliche Einstellungen\\Stichwort: Komponentendiagramm!}
\\In diesem Abschnitt sollen nun die Anforderungen und Regeln aus Archon herausgearbeitet werden. Die Anforderungen und Regeln werden, der Übersicht halber, in folgende Kategorien unterteilt: 
\begin{itemize}
	\item Generelle Regeln und Ziele des Spiels\label{generalRules}
	\item Spieleinstellungen und Start
	\item Das Spielbrett
	\item Das Kampfareal
	\item Die Figuren
\end{itemize}
Die folgende Abschnitte enthalten eine kurze Zusammenfassung des Spiels und seinen Inhalten und Regeln anhand der obigen Aufteilung. Anschließend wird kurz zusammengefasst, welche Eigenschaften eine Realisierung des Spiels inne haben muss, um das Spiel ausreichend widerzuspiegeln. 
\subsubsection{Generelle Regeln und Ziele des Spiels} 
Archon ist ein Spiel, dass auf einem 9 x 9 Schachbrett stattfindet. Ähnlich wie beim Schach gibt es zwei Parteien, Licht und Dunkelheit, die sich im Wettstreit gegenüber stehen.
Die Spieler ziehen dabei immer abwechselnd ihre Figuren auf dem Spielbrett, wobei kein Zug gepasst werden kann. Jeder Spieler beginnt mit 18 Figuren, die in acht verschiedene Typen gegliedert sind. Die Figuren von Licht und Dunkelheit sind vollständig unterschiedlich. Treffen zwei Figuren auf dem Spielbrett zusammen, stehen sie sich in einem Kampfareal gegenüber. Jede Figur hat dabei ihre eigenen Lebenspunkte und ihre eigene Angriffsstärke. Der Sieger dieses Kampfes bleibt auf dem Spielbrett, während der Verlierer aus dem Spiel genommen wird. Sollte der Kampf in einem Unentschieden ausgehen, werden beide Figuren aus dem Spiel genommen.
\\Das generelle Ziel des Spiels ist es alle 5 speziellen Machtfelder gleichzeitig mit eigenen Figuren zu besetzen.
Desweiteren kann das Spiel gewonnen werden, indem alle gegenerischen Figuren besiegt werden, oder die letzte Figur mit einem Gefängnis-Zauber belegt wird.
Ein Unentschieden tritt auf wenn das Spiel zu lange "passiv" ist, also wenn für mindestens zwölf Züge kein Kampf stattfindet und kein Zauber gewirkt wird.
\subsubsection{Das Spielbrett}
Das Spielbrett besteht aus 9x9 Feldern von drei Typen: 
\begin{itemize}
	\item Permanent weiße Felder
	\item Permanent schwarze Felder
	\item Felder, die zwischen Schwarz, vier Grün-Tönen und Weiß wechseln
\end{itemize}
Jeder Spieler hat einen Zug mit jeder Farbe, sodass die Farbe nach jedem Zug des zweiten Spielers wechselt. Ein Farbzyklus dauert damit zwölf Züge pro Seite: Sechs von Weiß zu Schwarz und Sechs zurück.
Die Farben werden fortan mit Zahlen von 1 bis 6 durchnummeriert, wobei 1 Weiß darstellt und 6 Schwarz.
Wenn die Licht-Seite das Spiel beginnt, startet der Farbzyklus bei Farbe 4 und wird dunkler. Beginnt die Dunkelheit-Seite, so startet der Zyklus bei Farbe 3 und wird heller.\\
Weiterhin gibt es die fünf Powerfelder, deren Einnahme die Siegbedingung darstellt. Diese Felder haben außerdem zwei spezielle Effekte auf Figuren, die darauf stehen. Zum einen können darauf stehende Figuren nicht Ziel eines Zaubers werden und zum Anderen werden die Figuren nach jedem eigenen Zug um einen gewissen Betrag geheilt. Diese fünf Powerfelder verteilen sich so, dass jeweils eines auf einem permanent schwarzen bzw. weißen Feld ist - dort wo der Zauberer bzw. die Zauberin ihre Ausgangsposition haben. Die anderen drei Felder sind auf farbwechselnden Feldern in der Mitte des Spielbretts verteilt.

\subsubsection{Das Kampfareal}
Das Kampfareal erscheint wenn zwei Figuren auf dem Spielbrett aufeinander treffen. Es stellt eine große Fläche dar, bei der von Zeit zu Zeit pflanzenähnliche Hindernisse erscheinen und wieder verschwinden. An den Seiten des Kampfareals werden die Lebenspunkte der kämpfenden Figuren in Form von Balken dargestellt.\\
Die Figuren können sich hier frei bewegen und haben, je nach Typ, unterschiedliche Eigenschaften, Attacken und Geschwindigkeiten.

\subsubsection{Die Figuren}

\subsubsection{Die Zauber}

\subsection{Anforderungsanalyse an neue Implementierung}
\label{subsec:neueImpl_analyse}\index{Analyse neue Implementierung}

\emph{Hier wird analysiert, welche Aspekte das Spiel von der Seite der technologisch gestellten Anforderungen erfüllen muss. Also: Es muss eine 3D- Engine geben, die irgendwie im Browser renderbar ist.\\ Es muss möglich sein, alle Spiel-Mechaniken mittels Server-Client-Kommunikation umzusetzen, etc.\\Stichwort: Komponentendiagramm!}

\section{Implementierung}
\label{sec:Umsetzung}\index{Implementierung}

\emph{Hier soll dann aufgezeigt werden, welche Anforderungen vom Spiel durch welche Technologie umgesetzt werden und nötige Bedingungen/grundsätzliche Ausschlüsse aufgezeigt werden.\\Hier, oder im nächsten Abschnitt muss klar gestellt werden, dass ein KI-Modus nicht implementiert wird!}

\subsection{Architektur}
\label{subsec:Architektur}\index{Architektur}

\emph{Aus dem Teil der Zusammenführung ergibt sich dann zwangsläufig eine Art Komponentenarchitektur, die hier verfeinert wird, bis zu dem Punkt, an dem eine Festlegung auf Frameworks und Technologien getätigt werden muss.}

\subsection{benötigte Technologien und Frameworks}
\label{subsec:Technologien}\index{Technologien}

\emph{Hier wird dann von den benötigten Technologien die Festlegung auf eine bestimmte Implementierung getroffen, also Frameworks, Programmiersprachen, Toolsets etc. festgelegt.}


\emph{Hier wird dann die Komponentenarchitektur vom vorherigen Teil weiter verfeinert, sodass evtl schon eine grobe Klassenarchitektur dabei herausspringen kann.}

\subsection{Hilfsmittel und Vereinfachungen}
\label{subsec:Hilfsmittel}\index{Hilfsmittel}

\emph{Hier würden Dinge zu lesen sein, wie die Benutzung von TypeScript, oder Browser-Beschränkungen etc, da diese nicht relevant für den eigentlichen Entwicklungsvorgang sind, aber dennoch nützlich sind und eben Vereinfachungen darstellen.}

\subsection{Schritte der Implementierung}
\label{subsec:Implementierung}\index{Implementierung}

\emph{Hier wird dann der Entwicklungsprozess kurz erläutert, also Vorgehen, wie Aufsetzen der Tool-Chain, dann Entwicklung einer ersten Darstellung (Frontend), um direkte Erfolge zu sehen, etc pp.}

\section{Resultate}
\label{sec:Resultate}\index{Resultate}

\emph{Hier soll dann ein Screenshot des Ergebnisses rein und erläutert werden, dass als nächste der Endstand mit seiner Architektur gezeigt wird und anschließend die Erfüllung aller Anforderungen sichergestellt wird. Als letztes (falls genug Zeit!) werden die (hoffentlich) programmierten Unit-Tests erwähnt, und deren Ergebnisse dargestellt.}

\subsection{fertige Architektur}
\label{subsec:f_architektur}

\emph{Die fertige Architektur kann und soll durchaus von der geplanten Abweichen und das schlussendliche Ergebnisse wird hier in Form von Diagrammen gezeigt, die dann einzeln erklärt werden.}

\subsection{Erfüllung der Anforderungen}
\label{subsec:erfullung_anforderungen}\index{Erfüllung}

\emph{Hier werden die Anforderungen aus dem Analyse-Teil aufgegriffen und mit der fertigen Anwendung und ihrer Architektur abgeglichen, also so was wie "die einzelnen Figuren und ihre Unterschiede, sind hier und hier da und da durch umgesetzt worden."}

\subsection{Überprüfung der Software mit Unit-Tests}
\label{subsec:unittests}\index{Unittests}

\emph{Hier wird dann die breite der Unit-Tests gezeigt, deren Anzahl und die Beschränkungen, also Code-Abdeckung. Anschließend ein Ergebnis-Log von einem Lauf auf dem finalen Stand.}


\section{Fazit}
\label{sec:Fazit}\index{Fazit}